\begin{problem}{Абсолютная величина}{стандартный ввод}{стандартный вывод}{1 секунда}{256 мегабайт}

Староста группы --- большой любитель математики, за прогулы вместо н-ок он ставит отрицательные числа. Пара закончилась и староста ушел в буфет, оставив журнал в аудитории. Стань тем самым учащимся с полной посещаемостью. 

Необходимо написать программу для замены всех отрицательных значений бинарного дерева их абсолютными величинами в процессе обхода дерева в ширину слева направо (BFS).

\InputFile
Первая строка содержит число N --- количество вершин в бинарном дереве. Каждая из последующих N строк содержит значение соответствующей вершины.
\vspace{1em}
$$0\le N\le 10^2$$
$$-10^{2}\le X_i, X_{i+1}, \ldots, X_N\le 10^2$$

\OutputFile
Если N = 0, выведите сообщение <<Дерево не содержит вершин>>.
Если N > 0, выведите все вершины бинарного дерева, используя метод обхода в ширину с лева направо (BFS).

\Examples

\begin{example}
\exmpfile{example.01}{example.01.a}%
\exmpfile{example.02}{example.02.a}%
\exmpfile{example.03}{example.03.a}%
\end{example}

\Note
Для решения используйте язык Java.

\end{problem}

